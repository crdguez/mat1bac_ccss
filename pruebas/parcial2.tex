\documentclass[addpoints,spanish, 12pt,a4paper]{exam}
%\documentclass[answers, spanish, 12pt,a4paper]{exam}
%\printanswers
\pointpoints{punto}{puntos}
\hpword{Puntos:}
\vpword{Puntos:}
\htword{Total}
\vtword{Total}
\hsword{Resultado:}
\hqword{Ejercicio:}
\vqword{Ejercicio:}

\usepackage[utf8]{inputenc}
\usepackage[spanish]{babel}
\usepackage{eurosym}
%\usepackage[spanish,es-lcroman, es-tabla, es-noshorthands]{babel}


\usepackage[margin=1in]{geometry}
\usepackage{amsmath,amssymb}
\usepackage{multicol}
\usepackage{yhmath}

\pointsinrightmargin % Para poner las puntuaciones a la derecha. Se puede cambiar. Si se comenta, sale a la izquierda.
\extrawidth{-2.4cm} %Un poquito más de margen por si ponemos textos largos.
\marginpointname{ \emph{\points}}

\usepackage{graphicx}
\graphicspath{{../img/}} 

\newcommand{\class}{1º Bachillerato CCSS}
\newcommand{\examdate}{\today}
\newcommand{\examnum}{Examen parcial 2ª evaluación}
\newcommand{\tipo}{A}


\newcommand{\timelimit}{50 minutos}

\renewcommand{\solutiontitle}{\noindent\textbf{Solución:}\enspace}


\pagestyle{head}
\firstpageheader{\includegraphics[width=0.2\columnwidth]{header_left}}{\textbf{Departamento de Matemáticas\linebreak \class}\linebreak \examnum}{\includegraphics[width=0.1\columnwidth]{header_right}}
\runningheader{\class}{\examnum}{Página \thepage\ of \numpages}
\runningheadrule


\begin{document}

\noindent
\begin{tabular*}{\textwidth}{l @{\extracolsep{\fill}} r @{\extracolsep{6pt}} }
\textbf{Nombre:} \makebox[3.5in]{\hrulefill} & \textbf{Fecha:}\makebox[1in]{\hrulefill} \\
 & \\
\textbf{Tiempo: \timelimit} & Tipo: \tipo 
\end{tabular*}
\rule[2ex]{\textwidth}{2pt}
Esta prueba tiene \numquestions\ ejercicios. La puntuación máxima es de \numpoints. 
La nota final de la prueba será la parte proporcional de la puntuación obtenida sobre la puntuación máxima. 

\begin{center}


\addpoints
 %\gradetable[h][questions]
	\pointtable[h][questions]
\end{center}

\noindent
\rule[2ex]{\textwidth}{2pt}

\begin{questions}

\question Una asociación de consumidores ha realizado una prueba sobre la duración de unas bombillas de una conocida marca. Ha mantenido encendidas ininterrumpidamente 100 bombillas  hasta que se han estropeado. Sus resultados han sido:



\begin{tabular}{cc}
\hline
   Duración en días &   Nº de bombillas \\
\hline
     $\left[36, 42\right)$ &     12      \\
     $\left[42, 48\right)$ &     28      \\
     $\left[48, 54\right)$ &     45      \\
     $\left[54, 60\right)$ &     15      \\
\hline
\end{tabular}

\begin{parts}
\part[1] Realizar una tabla de frecuencias con los datos que vayas a necesitar para resolver el ejercicio

\begin{solution}\\
\begin{tabular}{rrrrrrrrrr}
\hline
    &   lim\_inf &   lim\_sup &   x\_i &   f\_i &   F\_i &   h\_i &    H\_i &   x\_if\_i &   x\^{}2\_if\_i \\
\hline
  0 &        36 &        42 &    39 &    12 &    12 &  0.12 &   0.12 &      468 &      18252 \\
  1 &        42 &        48 &    45 &    28 &    40 &  0.28 &   0.4  &     1260 &      56700 \\
  2 &        48 &        54 &    51 &    45 &    85 &  0.45 &   0.85 &     2295 &     117045 \\
  3 &        54 &        60 &    57 &    15 &   100 &  0.15 &   1    &      855 &      48735 \\
\hline
  4 &       nan &       nan &   nan &   100 &   nan &  1    & nan    &     4878 &     240732 \\
\hline
\end{tabular}
\end{solution}
\part[1] Calcula la media y la varianza.
\begin{solution}
{'media': 48.78,
 'varianza': 27.83159999999998,
 'desviación típica': 5.27556632031102}
\end{solution}
\part[1] Indica razonadamente en qué intervalos se encuentra la moda y la mediana respectivamente. 
\begin{solution}
\end{solution}
\part[2] Calcula la mediana. Ayuda:
$$P_k=L_i + \frac{k\frac{N}{100}-F_{i-1}}{f_i}\cdot C_i$$
\begin{solution}
 ${'k': 50, 'N': 100.0, 'L_i': 48.0, 'f_i': 45.0, 'F_{i-1}': 40.0, 'C_i': 6.0}$ \\
 49.333333333333336
\end{solution}
\part[1] En el anuncio de televisión el fabricante asegura que sus bombillas duran más de 1200 horas. ¿Qué porcentaje de las bombillas no cumple lo enunciado?
\begin{solution}
${'valor': 50.0, 'N': 100.0, 'L_i': 48.0, 'f_i': 45.0, 'F_{i-1}': 40.0, 'C_i': 6.0}$\\
45.0
\end{solution}

\end{parts}


\question En una cofradía de pescadores, las capturas registradas de cierta variedad de pescados, en kilogramos,
y el precio de subasta en lonja, en euros kg, fueron los siguientes:\\
\begin{tabular}{|c||c|c|c|c|c|c|c|}
\hline 
x (Miles de kg) & 2 & 2.4 & 2.5 & 3 & 2.9 \\ 
\hline 
y (\euro /kg) & 1.8 & 1.68 & 1.65 & 1.32 & 1.44 \\ 
\hline 
\end{tabular} 
\addpoints % to omit double points count

\begin{parts}
\part[1] ¿Cuál es el precio medio registrado?

\begin{solution}\\
\begin{tabular}{rrrrrr}
\hline
    &     x &     y &      xy &     x2 &       y2 \\
\hline
  0 &  2    & 1.8   &  3.6    &  4     &  3.24    \\
  1 &  2.4  & 1.68  &  4.032  &  5.76  &  2.8224  \\
  2 &  2.5  & 1.65  &  4.125  &  6.25  &  2.7225  \\
  3 &  3    & 1.32  &  3.96   &  9     &  1.7424  \\
  4 &  2.9  & 1.44  &  4.176  &  8.41  &  2.0736  \\
\hline
  5 & 12.8  & 7.89  & 19.893  & 33.42  & 12.6009  \\
\hline
  6 &  2.56 & 1.578 &  3.9786 &  6.684 &  2.52018 \\
\hline
\end{tabular}\\

Precio medio = 1.578
\end{solution}
\part[2] Halla el coeficiente de correlación lineal e interprétalo
\begin{solution}
covarianza	-0.0610800000000005 \\
desvx	0.361109401705356 \\
desvy	0.17348198753761 \\
coefcorr -0.975002756630295 \\
\end{solution}
\part[1] Estima el precio que alcanzaría en lonja el kg de una especie si se pescasen 2600 kg 

\begin{solution}
$y=-0.46840490797546x+2.77711656441718$ \\
Valor estimado para 2.6: 1.5593 \euro / kg

\end{solution}


\end{parts}

\question En una bolsa, A, hay 2 bolas negras y 3 rojas. En otra bolsa, B, hay 3 bolas negras, 4 rojas y 3 verdes. Extraemos una bola de A y la introducimos en la bolsa B. Posteriormente, sacamos una bola de B:
%\noaddpoints % to omit double points count

\begin{parts}
\part[1] ¿Cuál es la probabilidad de que la segunda bola sea roja?
\begin{solution}
$\frac{8}{55}+\frac{2}{11}=\frac{23}{55}$ 
\end{solution}
\part[1] ¿Cuál es la probabilidad de que las dos bolas extraídas sean rojas?
\begin{solution}
$\frac{3}{11}$
\end{solution}
\end{parts}

\addpoints



\end{questions}

\end{document}
\grid
