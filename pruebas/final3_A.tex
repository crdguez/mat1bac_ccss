\documentclass[addpoints,spanish, 12pt,a4paper]{exam}
%\documentclass[answers, spanish, 12pt,a4paper]{exam}
%\printanswers
\pointpoints{punto}{puntos}
\hpword{Puntos:}
\vpword{Puntos:}
\htword{Total}
\vtword{Total}
\hsword{Resultado:}
\hqword{Ejercicio:}
\vqword{Ejercicio:}

\usepackage[utf8]{inputenc}
\usepackage[spanish]{babel}
\usepackage{eurosym}
%\usepackage[spanish,es-lcroman, es-tabla, es-noshorthands]{babel}


\usepackage[margin=1in]{geometry}
\usepackage{amsmath,amssymb}
\usepackage{multicol}
\usepackage{yhmath}
\usepackage{pdflscape}

\pointsinrightmargin % Para poner las puntuaciones a la derecha. Se puede cambiar. Si se comenta, sale a la izquierda.
\extrawidth{-2.4cm} %Un poquito más de margen por si ponemos textos largos.
\marginpointname{ \emph{\points}}

\usepackage{graphicx}
\graphicspath{{../img/}} 

\newcommand{\class}{1º Bachillerato CCSS}
\newcommand{\examdate}{\today}
\newcommand{\examnum}{Examen final 3ª Evaluación}
\newcommand{\tipo}{A}


\newcommand{\timelimit}{80 minutos}

\renewcommand{\solutiontitle}{\noindent\textbf{Solución:}\enspace}


\pagestyle{head}
\firstpageheader{\includegraphics[width=0.2\columnwidth]{header_left}}{\textbf{Departamento de Matemáticas\linebreak \class}\linebreak \examnum}{\includegraphics[width=0.1\columnwidth]{header_right}}
\runningheader{\class}{\examnum}{Página \thepage\ of \numpages}
\runningheadrule


\begin{document}

\noindent
\begin{tabular*}{\textwidth}{l @{\extracolsep{\fill}} r @{\extracolsep{6pt}} }
\textbf{Nombre:} \makebox[3.5in]{\hrulefill} & \textbf{Fecha:}\makebox[1in]{\hrulefill} \\
 & \\
\textbf{Tiempo: \timelimit} & Tipo: \tipo 
\end{tabular*}
\rule[2ex]{\textwidth}{2pt}
Esta prueba tiene \numquestions\ ejercicios. La puntuación máxima es de \numpoints. 
La nota final de la prueba será la parte proporcional de la puntuación obtenida sobre la puntuación máxima. 

\begin{center}


\addpoints
 %\gradetable[h][questions]
	\pointtable[h][questions]
\end{center}

\noindent
\rule[2ex]{\textwidth}{2pt}

\begin{questions}
\question Dada la función $f(x)=\dfrac{x + 2}{- x + 1}$:

        \begin{parts}  \part[2] Calcula su inversa \begin{solution}   $f^{-1}(x)=\frac{x - 2}{x + 1}$  \end{solution}
        \part[2] Comprueba que son inversas \begin{solution}   $f^{-1} \circ f(x)=\frac{-2 + \frac{x + 2}{- x + 1}}{1 + \frac{x + 2}{- x + 1}}=x$ \end{solution}
        \end{parts}
        
\question Calcula los siguientes límites:
        \begin{parts} \part[1] $$\lim_{x \to -2}\left(\frac{2 x^{2} + 7 x + 6}{x^{3} + 3 x^{2} + 3 x + 2}\right)$$  \begin{solution}   $- \frac{1}{3}$   \end{solution} \part[2] $$\lim_{x \to -1}\left(\frac{x^{3} + 1}{x^{2} + 2 x + 1}\right)$$  \begin{solution}   No existe el límite   \end{solution} \part[2] $$\lim_{x \to \infty} \left(\frac{3 x - 1}{3 x - 2}\right)^{2 x}$$  \begin{solution}   $e^{\frac{2}{3}}$   \end{solution}
        \end{parts}


\question Dada la función: $$f(x)=\begin{cases} k + x & \text{si}\: x \leq 0 \\x^{2} - 1 & \text{si}\: x>0 \end{cases}$$
        \begin{parts} \part[2] Calcula el valor de k para que sea continua \begin{solution}   $\left\{-1\right\}$   \end{solution} 
        \end{parts}
        
\question Calcula las siguientes derivadas:
        \begin{parts} 
        \part[1] $$y = x^{5} + 3 x^{3} - 2 x^{2} - x + 3$$  \begin{solution}   $y'=5 x^{4} + 9 x^{2} - 4 x - 1$   \end{solution}
        \part[1] $$y = \left(3 x + 2\right)^{3}$$  \begin{solution}   $y'=9 \left(3 x + 2\right)^{2}$   \end{solution}
        \part[1] $$y = \left(2 x + 3\right) \cdot \left(x^{2} - 2 x + 100\right)$$  \begin{solution}   $y'=6 x^{2} - 2 x + 194$   \end{solution}
        \part[1] $$y = \ln{\left(x^{2} - 2 x \right)}$$  
        \begin{solution}   $y'=\frac{2 \left(x - 1\right)}{x \left(x - 2\right)}$   
        \end{solution} 
        
        \end{parts}
        
\question Dada la función: $$f(x)=x^3-27x$$ Calcula:
        \begin{parts} 
        \part[1] Los puntos singulares \begin{solution}   $\left\{-3, 3\right\}$   \end{solution}
        \part[2] Los intervalos de crecimiento  \begin{solution}   $\left(-\infty, -3\right) \cup \left(3, \infty\right)$   \end{solution}
        \part[1] Las asíntotas  \begin{solution}   No tiene   \end{solution}
       
        
        \end{parts}




        

        

\addpoints




\end{questions}


\end{document}

