\documentclass[addpoints,spanish, 12pt,a4paper]{exam}
%\documentclass[answers, spanish, 12pt,a4paper]{exam}
\printanswers
\pointpoints{punto}{puntos}
\hpword{Puntos:}
\vpword{Puntos:}
\htword{Total}
\vtword{Total}
\hsword{Resultado:}
\hqword{Ejercicio:}
\vqword{Ejercicio:}

\usepackage[utf8]{inputenc}
\usepackage[spanish]{babel}
\usepackage{eurosym}
%\usepackage[spanish,es-lcroman, es-tabla, es-noshorthands]{babel}


\usepackage[margin=1in]{geometry}
\usepackage{amsmath,amssymb}
\usepackage{multicol}
\usepackage{yhmath}

\pointsinrightmargin % Para poner las puntuaciones a la derecha. Se puede cambiar. Si se comenta, sale a la izquierda.
\extrawidth{-2.4cm} %Un poquito más de margen por si ponemos textos largos.
\marginpointname{ \emph{\points}}

\usepackage{graphicx}
\graphicspath{{../img/}} 

\newcommand{\class}{1º Bachillerato CCSS}
\newcommand{\examdate}{\today}
\newcommand{\examnum}{Examen de números reales}
\newcommand{\tipo}{A}


\newcommand{\timelimit}{80 minutos}

\renewcommand{\solutiontitle}{\noindent\textbf{Solución:}\enspace}


\pagestyle{head}
\firstpageheader{\includegraphics[width=0.2\columnwidth]{header_left}}{\textbf{Departamento de Matemáticas\linebreak \class}\linebreak \examnum}{\includegraphics[width=0.1\columnwidth]{header_right}}
\runningheader{\class}{\examnum}{Página \thepage\ of \numpages}
\runningheadrule


\begin{document}

\noindent
\begin{tabular*}{\textwidth}{l @{\extracolsep{\fill}} r @{\extracolsep{6pt}} }
\textbf{Nombre:} \makebox[3.5in]{\hrulefill} & \textbf{Fecha:}\makebox[1in]{\hrulefill} \\
 & \\
\textbf{Tiempo: \timelimit} & Tipo: \tipo 
\end{tabular*}
\rule[2ex]{\textwidth}{2pt}
Esta prueba tiene \numquestions\ ejercicios. La puntuación máxima es de \numpoints. 
La nota final de la prueba será la parte proporcional de la puntuación obtenida sobre la puntuación máxima. 

\begin{center}


\addpoints
 %\gradetable[h][questions]
	\pointtable[h][questions]
\end{center}

\noindent
\rule[2ex]{\textwidth}{2pt}

\begin{questions}

\question[2] Indica a cuáles de los conjuntos
$\mathbb{N}$, $\mathbb{Z}$, $\mathbb{Q}$, $\mathbb{R}$ pertenecen cada uno de los siguientes números:
\begin{center}
\begin{tabular}{|c |c |c |c |c|}\hline
&$\mathbb{N}$& $\mathbb{Z}$& $\mathbb{Q}$&$\mathbb{R}$\\ 
\hline
$5$&&&&\\
\hline
$-7$&&&&\\
\hline
$0,23$&&&&\\
\hline
$\sqrt{\frac{18}{2}}$&&&&\\
\hline
$-\sqrt{3}$&&&&\\
\hline
$\sqrt[3]{-5}$&&&&\\
\hline
$4,\wideparen{7}$&&&&\\
\hline
$\frac{-\pi}{2}$&&&&\\
\hline
$-\sqrt{25}$&&&&\\
\hline
$\sqrt{-4}$&&&&\\
\hline
\end{tabular}

\end{center}

\begin{solution}
\begin{tabular}{|c |c |c |c |c|}\hline
&$\mathbb{N}$& $\mathbb{Z}$& $\mathbb{Q}$&$\mathbb{R}$\\ 
\hline
$5$&X&X&X&X\\
\hline
$-7$&&X&X&X\\
\hline
$0,23$&&&X&X\\
\hline
$\sqrt{\frac{18}{2}}$&X&X&X&X\\
\hline
$-\sqrt{3}$&&&&X\\
\hline
$\sqrt[3]{-5}$&&&&X\\
\hline
$4,\wideparen{7}$&&&X&X\\
\hline
$\frac{-\pi}{2}$&&&&X\\
\hline
$-\sqrt{25}$&&X&X&X\\
\hline
$\sqrt{-4}$&&&&\\
\hline
\end{tabular}
\end{solution}

\addpoints



\question[2] Calcula un número que restado con el doble de su raíz cuadrada nos de 15.
\addpoints % to omit double points count


\begin{solution}
 	$- 2 \sqrt{x} + x - 15 = 0\to \left\{25\right\}$ 
\end{solution}

\question[2] Efectúa la siguiente operación, dando el resultado en notación científica y con la mantisa redondeada a las centésimas. Da, en notación científica también, una cota del error absoluto producido en el redondeo.
$$\frac{5.12\cdot {10}^3 \cdot 4.2\cdot {10}^7}{1.8 \cdot {10}^{15}}$$
\addpoints % to omit double points count


\begin{solution}
 	$\approx 0.000119466666666667 \approx 1.19\cdot {10}^{-4} < 0.5 \cdot 0.01 \cdot {10}^{-4} = 5\cdot {10}^{-7}$ 
 
\end{solution}

\question Expresa en forma de intervalo:
%\noaddpoints % to omit double points count

\begin{parts}
\part[1] $\left| {x - 4} \right|< 5$ 
\begin{solution}
$\left(-1, 9\right)$ 
\end{solution}
\part[1] $\left| {x +3 } \right|\geqslant 2$
\begin{solution}
$\left(-\infty, -5\right] \cup \left[.1, \infty\right) $
\end{solution}
\end{parts}

\addpoints


\question Opera y simplifica:
%\noaddpoints % to omit double points count

\begin{parts}
\part[1] \[4\sqrt{20}-3\sqrt{45}+11\sqrt{125}-20\sqrt{5}\]
\begin{solution}
$=4\cdot2\sqrt{5}-3\cdot3\sqrt{5}+11\cdot5\sqrt{5}-20\sqrt{5}=\left(8-9+55-20\right)\sqrt{5}=34\sqrt{5}$
\end{solution}

\part[1] $$\left( {\sqrt[4]{a^3} \frac{1}{a}} \right):\left( {a\sqrt {a} } \right) $$
\begin{solution}
$a^{-\frac{7}{4}}$
\end{solution}

\part[1] $$\sqrt{8ab}\cdot\sqrt[3]{a^2b}$$
\begin{solution}
$=\sqrt[6]{2^9a^3b^3\cdot a^4b^2}=2a\sqrt[6]{2^3ab^5} $
\end{solution}




\end{parts}
\addpoints

\question Racionaliza y simplifica:
%\noaddpoints % to omit double points count
\begin{parts}
\part[1] \[\dfrac{10}{2\sqrt{3}-\sqrt{2}}\]
\begin{solution}
$=\dfrac{10\cdot\left(2\sqrt{3}+\sqrt{2}\right)}{\left(2\sqrt{3}-\sqrt{2}\right)\cdot\left(2\sqrt{3}+\sqrt{2}\right)}=\dfrac{10\cdot\left(2\sqrt{3}+\sqrt{2}\right)}{4\cdot3-2}=2\sqrt{3}+\sqrt{2}$
\end{solution}


\part[1] \[\frac{4 + \sqrt {6} }{2\sqrt {3 }}\]
\begin{solution}
$=\dfrac{4\sqrt{3}+\sqrt{6}\cdot\sqrt{3}}{2\sqrt{3}\sqrt{3}}=\dfrac{4\sqrt{3}+3\sqrt{2}}{6}$
\end{solution}



\end{parts}
\addpoints


\question Calcula x, aplicando la definición de logaritmo:

\begin{parts}

\part[1] ${\log _2} 0.5 = x$ \begin{solution} $2^x=2^{-1}\to x= -1$ \end{solution}

\part[1] ${\log _4} x = -\frac{1}{2}$ \begin{solution} $ 4^{-\frac{1}{2}} = x \to x = \frac{1}{\sqrt{4}} \to x = \frac{1}{2} $\end{solution}

\part[1] ${\log _5}\sqrt{125}=x$ \begin{solution} $5^x=5^{\frac{3}{2}}\to x=\frac{3}{2} $ \end{solution}

\part[1] ${\log _x}36 = 4 $ \begin{solution} $x^4=36 \to x=\sqrt[4]{6^2}\to x = \sqrt{6} $ \end{solution}
\end{parts}



\addpoints


\question Calcula:
%\noaddpoints % to omit double points count

\begin{parts}


\part[1] \[\log_3 \frac{1}{9} -  \log_5 0,2 +\log_6 \frac{1}{36} - \log_2 0,5\]
\begin{solution}
$=-2-\left(-1\right)+\left(-2\right)-\left(-1\right)=-2$
\end{solution}




\end{parts}

\addpoints


\question Calcula sabiendo que $\log a = 2.5$ y $\log b = -1.2$ :
%\noaddpoints % to omit double points count

\begin{parts}
\part[1] \[log\,\frac{\sqrt[5]{{a^2}{b^4}}  }{\sqrt[3]{{a^5}b} }\]
\begin{solution}
$=\frac{1}{5}\log {a^2b^4} - \frac{1}{3}\log {a^5b}=\frac{1}{5}\left[2\log a + 4\log b \right]-\frac{1}{3}\left[5\log a +\log b \right]\approx-3.72666666666667$
\end{solution}

\end{parts}

\addpoints

\end{questions}

\end{document}
\grid
