\documentclass[addpoints,spanish, 12pt,a4paper]{exam}
%\documentclass[answers, spanish, 12pt,a4paper]{exam}
%\printanswers
\pointpoints{punto}{puntos}
\hpword{Puntos:}
\vpword{Puntos:}
\htword{Total}
\vtword{Total}
\hsword{Resultado:}
\hqword{Ejercicio:}
\vqword{Ejercicio:}

\usepackage[utf8]{inputenc}
\usepackage[spanish]{babel}
\usepackage{eurosym}
%\usepackage[spanish,es-lcroman, es-tabla, es-noshorthands]{babel}


\usepackage[margin=1in]{geometry}
\usepackage{amsmath,amssymb}
\usepackage{multicol}
\usepackage{yhmath}

\pointsinrightmargin % Para poner las puntuaciones a la derecha. Se puede cambiar. Si se comenta, sale a la izquierda.
\extrawidth{-2.4cm} %Un poquito más de margen por si ponemos textos largos.
\marginpointname{ \emph{\points}}

\usepackage{graphicx}
\graphicspath{{../img/}} 

\newcommand{\class}{1º Bachillerato CCSS}
\newcommand{\examdate}{\today}
\newcommand{\examnum}{Recuperación 1ª Evaluación}
\newcommand{\tipo}{B}


\newcommand{\timelimit}{45 minutos}

\renewcommand{\solutiontitle}{\noindent\textbf{Solución:}\enspace}


\pagestyle{head}
\firstpageheader{\includegraphics[width=0.2\columnwidth]{header_left}}{\textbf{Departamento de Matemáticas\linebreak \class}\linebreak \examnum}{\includegraphics[width=0.1\columnwidth]{header_right}}
\runningheader{\class}{\examnum}{Página \thepage\ of \numpages}
\runningheadrule


\begin{document}

\noindent
\begin{tabular*}{\textwidth}{l @{\extracolsep{\fill}} r @{\extracolsep{6pt}} }
\textbf{Nombre:} \makebox[3.5in]{\hrulefill} & \textbf{Fecha:}\makebox[1in]{\hrulefill} \\
 & \\
\textbf{Tiempo: \timelimit} & Tipo: \tipo 
\end{tabular*}
\rule[2ex]{\textwidth}{2pt}
Esta prueba tiene \numquestions\ ejercicios. La puntuación máxima es de \numpoints. 
La nota final de la prueba será la parte proporcional de la puntuación obtenida sobre la puntuación máxima. 

\begin{center}


\addpoints
 %\gradetable[h][questions]
	\pointtable[h][questions]
\end{center}

\noindent
\rule[2ex]{\textwidth}{2pt}

\begin{questions}




\question[2] El perímetro de un triángulo rectángulo es 90 m y el cateto mayor tiene 3 m menos que la hipotenusa. Halla los tres lados del triángulo.
\addpoints % to omit double points count


\begin{solution}
 	$x + 2 y - 87 = 0\land x^{2} + y^{2} - \left(y + 3\right)^{2} = 0\to \left [ \left ( -18, \quad \frac{105}{2}\right ), \quad \left ( 15, \quad 36\right )\right ]$ 
\end{solution}




\question Expresa en forma de intervalo:
%\noaddpoints % to omit double points count

\begin{parts}
\part[1] $\left| {x +3 } \right|\geqslant 1$
\begin{solution}
$\left(-\infty, -4\right] \cup \left[-2, \infty\right) $
\end{solution}
\end{parts}

\addpoints


\question Opera con radicales y simplifica:
%\noaddpoints % to omit double points count

\begin{parts}
\part[1]  $ 4\sqrt {3125}  + 2\sqrt {20}  - 30\sqrt {45} $  \begin{solution}  $ 14 \sqrt{5} $  \end{solution}

\end{parts}
\addpoints

\question Racionaliza y simplifica:
%\noaddpoints % to omit double points count
\begin{parts}
\part[1] \[\dfrac{10}{2\sqrt{3}-\sqrt{2}}\]
\begin{solution}
$=\dfrac{10\cdot\left(2\sqrt{3}+\sqrt{2}\right)}{\left(2\sqrt{3}-\sqrt{2}\right)\cdot\left(2\sqrt{3}+\sqrt{2}\right)}=\dfrac{10\cdot\left(2\sqrt{3}+\sqrt{2}\right)}{4\cdot3-2}=2\sqrt{3}+\sqrt{2}$
\end{solution}





\end{parts}
\addpoints




\question Calcula aplicando la definición de logaritmo:
%\noaddpoints % to omit double points count

\begin{parts}


\part[1] \[\log_3 \frac{1}{9} -  \log_5 0,2 +\log_6 \frac{1}{36} - \log_2 0,5\]
\begin{solution}
$=-2-\left(-1\right)+\left(-2\right)-\left(-1\right)=-2$
\end{solution}




\end{parts}

\addpoints


\question Calcula sabiendo que $\log a = 2.5$ y $\log b = -1.2$ :
%\noaddpoints % to omit double points count

\begin{parts}
\part[1] \[log\,\frac{\sqrt[5]{{a^2}{b^4}}  }{\sqrt[3]{{a^5}b} }\]
\begin{solution}
$=\frac{1}{5}\log {a^2b^4} - \frac{1}{3}\log {a^5b}=\frac{1}{5}\left[2\log a + 4\log b \right]-\frac{1}{3}\left[5\log a +\log b \right]\approx-3.72666666666667$
\end{solution}

\end{parts}

\addpoints

\question Resuelve:
 
        \begin{parts} \part[1]  $ 3 + x < 5 - x\cdot( {x - 2} ) $  \begin{solution}  $\left(-1, 2\right) $  \end{solution}
        \end{parts}
        

        \question Discute el tipo de sistema y resuelve si es posible:

        \begin{parts} 
        \part[1]  $ \left\{\begin{matrix}x - 2y + z = 13\\ 3x - 4y + 2z = 1\\ 2x - 2y + z = 0\\ \end{matrix}\right. $  \begin{solution}  $ \left[\begin{matrix}1 & -2 & 1 & 13\\0 & 2 & -1 & -38\\0 & 0 & 0 & 12\end{matrix}\right] \rightarrow  \\ \left [ \right ] $  \end{solution}
        

        \end{parts}



\end{questions}

\end{document}
\grid
