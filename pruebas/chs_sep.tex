\documentclass[addpoints,spanish, 12pt,a4paper]{exam}
%\documentclass[answers, spanish, 12pt,a4paper]{exam}
\printanswers
\pointpoints{punto}{puntos}
\hpword{Puntos:}
\vpword{Puntos:}
\htword{Total}
\vtword{Total}
\hsword{Resultado:}
\hqword{Ejercicio:}
\vqword{Ejercicio:}

\usepackage[utf8]{inputenc}
\usepackage[spanish]{babel}
\usepackage{eurosym}
%\usepackage[spanish,es-lcroman, es-tabla, es-noshorthands]{babel}


\usepackage[margin=1in]{geometry}
\usepackage{amsmath,amssymb}
\usepackage{multicol}
\usepackage{yhmath}
\usepackage{pdflscape}

\pointsinrightmargin % Para poner las puntuaciones a la derecha. Se puede cambiar. Si se comenta, sale a la izquierda.
\extrawidth{-2.4cm} %Un poquito más de margen por si ponemos textos largos.
\marginpointname{ \emph{\points}}

\usepackage{graphicx}
\graphicspath{{../img/}} 

\newcommand{\class}{1º Bachillerato CCSS}
\newcommand{\examdate}{\today}
\newcommand{\examnum}{Extraordinario de septiembre}
\newcommand{\tipo}{A}


\newcommand{\timelimit}{80 minutos}

\renewcommand{\solutiontitle}{\noindent\textbf{Solución:}\enspace}


\pagestyle{head}
\firstpageheader{\includegraphics[width=0.2\columnwidth]{header_left}}{\textbf{Departamento de Matemáticas\linebreak \class}\linebreak \examnum}{\includegraphics[width=0.1\columnwidth]{header_right}}
\runningheader{\class}{\examnum}{Página \thepage\ of \numpages}
\runningheadrule


\begin{document}

\noindent
\begin{tabular*}{\textwidth}{l @{\extracolsep{\fill}} r @{\extracolsep{6pt}} }
\textbf{Fecha:}\makebox[1in]{\hrulefill} \textbf{Nombre:} \makebox[.75in]{\hrulefill} @@alumno \makebox[.75in]{\hrulefill} \\
 & \\
\textbf{Tiempo: \timelimit} & Tipo: \tipo 
\end{tabular*}
\rule[2ex]{\textwidth}{2pt}
Esta prueba tiene \numquestions\ ejercicios. La puntuación máxima es de \numpoints. 
La nota final de la prueba será la parte proporcional de la puntuación obtenida sobre la puntuación máxima. 

\begin{center}


\addpoints
 %\gradetable[h][questions]
	\pointtable[h][questions]
\end{center}

\noindent
\rule[2ex]{\textwidth}{2pt}

\begin{questions}

\question[2] Opera:
		\begin{parts}  
				\part $$\frac{{\sqrt {2}  - \sqrt {3} }}{{\sqrt{ 2}  + \sqrt {3} }}$$ \begin{solution}   $- \left(- \sqrt{3} + \sqrt{2}\right)^{2}$  \end{solution}
			\part $$\log{\frac{1}{10}}+\log_{2}{\sqrt{32}}-\log_2{\frac{1}{4}}$$ \begin{solution}   $-1+\frac{5}{2}-(-2)=\frac{7}{2}$  \end{solution}
		        \part $$\log{\left(10\cdot k^3\right)}$$ sabiendo que $ \log k = 1.1$ \begin{solution}   $1+3\cdot 1.1 = 4.3$ \end{solution}
		        \end{parts}


\question[2] Calcula: $$ \left\{\begin{matrix}\dfrac{{x - 1}}{3} - \dfrac{{x + 3}}{2} \leq x\\ \dfrac{{4x - 2}}{4} - \dfrac{{x - 1}}{3} \geq x\end{matrix}\right.$$ \begin{solution}   $\left[- \dfrac{11}{7}, - \dfrac{1}{2}\right]$ \end{solution}

\question[2] Dentro de dos años, la raíz cuadrada de la edad de Sergio coincidirá con la que tenga su hija Julia. Calcula la edad de cada uno, sabiendo que, ahora, la suma de las edades es 52.
\begin{solution}   47 y 5 \end{solution}

\question La temperatura media en los meses de invierno en varias ciudades y el gasto medio por habitante en
calefacción ha sido:\\
\\
\begin{tabular}{|c||c|c|c|c|c|c|}
\hline 
Temperatura (ºC) & 10 & 12 & 14 & 16  \\ 
\hline 
Gasto (\euro ) & 150 & 120 & 102 & 90  \\ 
\hline 
\end{tabular} \\
\begin{parts}
\part[1] Halla el coeficiente de correlación lineal 
\begin{solution}
\begin{tabular}{rrrrrr}
\hline
    &   x &     y &   xy &   x2 &    y2 \\
\hline
  0 &  10 & 150   & 1500 &  100 & 22500 \\
  1 &  12 & 120   & 1440 &  144 & 14400 \\
  2 &  14 & 102   & 1428 &  196 & 10404 \\
  3 &  16 &  90   & 1440 &  256 &  8100 \\
  \hline
  4 &  52 & 462   & 5808 &  696 & 55404 \\
  \hline
  5 &  13 & 115.5 & 1452 &  174 & 13851 \\
\hline
\end{tabular}
\\

covarianza	-49.5 \\
desvx	2.23606797749979 \\
desvy	22.599778759979046 \\
coefcorr -0.9795260923726159 \\
\end{solution}
\part[1] Estima, razonadamente, el gasto medio por habitante de una ciudad si la temperatura media hubiera sido de 11ºC. ¿Es fiable la estimación obtenida? 

\begin{solution}
$y = -9.9x + 244.2$ \\ Valor estimado para 11: 135.3 \euro 
\end{solution}
\end{parts}

\question Un jugador de baloncesto tiene un porcentaje de acierto en tiros de 3 del 40 \%. Si tira seis
veces:
    

        \begin{parts} 
        
        \part[1] Calcula la probabilidad de que enceste 4  \begin{solution}   $P{\left(X = 4 \right)}=0.1382$   \end{solution} \part[1] Calcula la probabilidad de que enceste al menos 1  \begin{solution}   $P{\left(X \geq 1 \right)}=1-P{\left(X = 0 \right)}=0.9533$   \end{solution} \part[1] Calcula la probabilidad de que enceste más de 3 si ha fallado los dos primeros \begin{solution}   $P{\left(X' = 4 \right)}=0.0256$   \end{solution}
        \end{parts}



\question La duración media de un televisor es de 10 años con una desviación típica igual a 2 años. Si la
vida útil del electrodoméstico se distribuye normalmente, halla la probabilidad de que al comprar
un televisor: 
    
        
        \begin{parts}  \part[1]  Este dure más de 12 años  \begin{solution}   $P{\left(X \geq 12 \right)}=0.158655253931457$   \end{solution} \part[1]  Este dure entre 8 y 12 años  \begin{solution}   $P{\left(X \geq 8 \wedge X \leq 12 \right)}=0.682689492137086$   \end{solution}
        \end{parts}

\question Dadas las funciones $$f(x)= x^2+5$$ $$g(x)= \frac{{x - 1}}{{x + 3}}$$ 
%y $h(x)= \sqrt{x}$
. Calcula:

	\begin{parts}  \part[1] $g$ compuesta con $f$. Es decir, $f \circ g$  \begin{solution} $f{\left (g{\left (x \right( )} \right )}=\frac{\left(x - 1\right)^{2}}{\left(x + 3\right)^{2}} + 5$
	\end{solution}
	        \part[1] La inversa de $g$. Es decir, $g^{-1}(x)$ \begin{solution}   $g^{-1}(x)=- \frac{3 x + 1}{x - 1}$ \end{solution}
	        \end{parts}

\question Calcula:

	\begin{parts}  
\part[1] $$\lim_{x \to 0}\left(\frac{2 x^{3} + 6 x^{2} - 3 x}{2 x^{2} + 5 x}\right)$$ \begin{solution}   $- \frac{3}{5}$ \end{solution}	
	\part[1] $$\lim_{x \to \infty} \left(\frac{x^{2} + 3}{3 x^{2} - 5}\right)^{\frac{x^{2}}{2 - x}}$$  \begin{solution} $\infty$
	\end{solution}
	        
	        \end{parts}

        
\question Dada $f(x)=x^4-3x^2+5$. Calcula: 

        \begin{parts}
        \part[1] Dominio de $f(x)$ y su función derivada  \begin{solution}   $\mathbb{R}$ y $f'(x)=4x^3-6x$ \end{solution} 
        \part[1] La recta tangente a la función por el punto P(-1,3)  \begin{solution}   $y=2x+5$          
         \end{solution}
         \end{parts}

%
%\question Dada la función $f(x)=\dfrac{x + 2}{- x + 1}$:
%
%        \begin{parts}  \part[2] Calcula su inversa \begin{solution}   $f^{-1}(x)=\frac{x - 2}{x + 1}$  \end{solution}
%        \part[2] Comprueba que son inversas \begin{solution}   $f^{-1} \circ f(x)=\frac{-2 + \frac{x + 2}{- x + 1}}{1 + \frac{x + 2}{- x + 1}}=x$ \end{solution}
%        \end{parts}
%        
%\question Calcula los siguientes límites:
%        \begin{parts} \part[1] $$\lim_{x \to -2}\left(\frac{2 x^{2} + 7 x + 6}{x^{3} + 3 x^{2} + 3 x + 2}\right)$$  \begin{solution}   $- \frac{1}{3}$   \end{solution} \part[2] $$\lim_{x \to -1}\left(\frac{x^{3} + 1}{x^{2} + 2 x + 1}\right)$$  \begin{solution}   No existe el límite   \end{solution} \part[2] $$\lim_{x \to \infty} \left(\frac{3 x - 1}{3 x - 2}\right)^{2 x}$$  \begin{solution}   $e^{\frac{2}{3}}$   \end{solution}
%        \end{parts}
%
%
%\question Dada la función: $$f(x)=\begin{cases} k + x & \text{si}\: x \leq 0 \\x^{2} - 1 & \text{si}\: x>0 \end{cases}$$
%        \begin{parts} \part[2] Calcula el valor de k para que sea continua \begin{solution}   $\left\{-1\right\}$   \end{solution} 
%        \end{parts}
%        
%\question Calcula las siguientes derivadas:
%        \begin{parts} 
%        \part[1] $$y = x^{5} + 3 x^{3} - 2 x^{2} - x + 3$$  \begin{solution}   $y'=5 x^{4} + 9 x^{2} - 4 x - 1$   \end{solution}
%        \part[1] $$y = \left(3 x + 2\right)^{3}$$  \begin{solution}   $y'=9 \left(3 x + 2\right)^{2}$   \end{solution}
%        \part[1] $$y = \left(2 x + 3\right) \cdot \left(x^{2} - 2 x + 100\right)$$  \begin{solution}   $y'=6 x^{2} - 2 x + 194$   \end{solution}
%        \part[1] $$y = \ln{\left(x^{2} - 2 x \right)}$$  
%        \begin{solution}   $y'=\frac{2 \left(x - 1\right)}{x \left(x - 2\right)}$   
%        \end{solution} 
%        
%        \end{parts}
%        
%\question Dada la función: $$f(x)=x^3-27x$$ Calcula:
%        \begin{parts} 
%        \part[1] Los puntos singulares \begin{solution}   $\left\{-3, 3\right\}$   \end{solution}
%        \part[2] Los intervalos de crecimiento  \begin{solution}   $\left(-\infty, -3\right) \cup \left(3, \infty\right)$   \end{solution}
%        \part[1] Las asíntotas  \begin{solution}   No tiene   \end{solution}
%       
%        
%        \end{parts}
%
%
%
%
        

        

\addpoints




\end{questions}

    \newgeometry{left=1 cm,bottom=2cm}
\begin{landscape}
\begin{table}[]
\Large
\centering
\caption{Extracto de tabla de probabilidades de la \textbf{normal estándar $Z(0,1)$}}
\label{my-label}

\begin{tabular}{l|llllllllll}
z   & 0       & 0,01    & 0,02    & 0,03    & 0,04    & 0,05    & 0,06    & 0,07    & 0,08    & 0,09    \\
\hline
0   & 0,5     & 0,50399 & 0,50798 & 0,51197 & 0,51595 & 0,51994 & 0,52392 & 0,5279  & 0,53188 & 0,53586 \\
0,1 & 0,53983 & 0,5438  & 0,54776 & 0,55172 & 0,55567 & 0,55962 & 0,56356 & 0,56749 & 0,57142 & 0,57535 \\
0,2 & 0,57926 & 0,58317 & 0,58706 & 0,59095 & 0,59483 & 0,59871 & 0,60257 & 0,60642 & 0,61026 & 0,61409 \\
0,3 & 0,61791 & 0,62172 & 0,62552 & 0,6293  & 0,63307 & 0,63683 & 0,64058 & 0,64431 & 0,64803 & 0,65173 \\
0,4 & 0,65542 & 0,6591  & 0,66276 & 0,6664  & 0,67003 & 0,67364 & 0,67724 & 0,68082 & 0,68439 & 0,68793 \\
0,5 & 0,69146 & 0,69497 & 0,69847 & 0,70194 & 0,7054  & 0,70884 & 0,71226 & 0,71566 & 0,71904 & 0,7224  \\
0,6 & 0,72575 & 0,72907 & 0,73237 & 0,73565 & 0,73891 & 0,74215 & 0,74537 & 0,74857 & 0,75175 & 0,7549  \\
0,7 & 0,75804 & 0,76115 & 0,76424 & 0,7673  & 0,77035 & 0,77337 & 0,77637 & 0,77935 & 0,7823  & 0,78524 \\
0,8 & 0,78814 & 0,79103 & 0,79389 & 0,79673 & 0,79955 & 0,80234 & 0,80511 & 0,80785 & 0,81057 & 0,81327 \\
0,9 & 0,81594 & 0,81859 & 0,82121 & 0,82381 & 0,82639 & 0,82894 & 0,83147 & 0,83398 & 0,83646 & 0,83891 \\
1   & 0,84134 & 0,84375 & 0,84614 & 0,84849 & 0,85083 & 0,85314 & 0,85543 & 0,85769 & 0,85993 & 0,86214 \\
%1,1 & 0,86433 & 0,8665  & 0,86864 & 0,87076 & 0,87286 & 0,87493 & 0,87698 & 0,879   & 0,881   & 0,88298 \\
%1,2 & 0,88493 & 0,88686 & 0,88877 & 0,89065 & 0,89251 & 0,89435 & 0,89617 & 0,89796 & 0,89973 & 0,90147 \\
%1,3 & 0,9032  & 0,9049  & 0,90658 & 0,90824 & 0,90988 & 0,91149 & 0,91309 & 0,91466 & 0,91621 & 0,91774 \\
%1,4 & 0,91924 & 0,92073 & 0,9222  & 0,92364 & 0,92507 & 0,92647 & 0,92785 & 0,92922 & 0,93056 & 0,93189 \\
%1,5 & 0,93319 & 0,93448 & 0,93574 & 0,93699 & 0,93822 & 0,93943 & 0,94062 & 0,94179 & 0,94295 & 0,94408 \\
%1,6 & 0,9452  & 0,9463  & 0,94738 & 0,94845 & 0,9495  & 0,95053 & 0,95154 & 0,95254 & 0,95352 & 0,95449 \\
%1,7 & 0,95543 & 0,95637 & 0,95728 & 0,95818 & 0,95907 & 0,95994 & 0,9608  & 0,96164 & 0,96246 & 0,96327 \\
%1,8 & 0,96407 & 0,96485 & 0,96562 & 0,96638 & 0,96712 & 0,96784 & 0,96856 & 0,96926 & 0,96995 & 0,97062 \\
%1,9 & 0,97128 & 0,97193 & 0,97257 & 0,9732  & 0,97381 & 0,97441 & 0,975   & 0,97558 & 0,97615 & 0,9767  \\
%2   & 0,97725 & 0,97778 & 0,97831 & 0,97882 & 0,97932 & 0,97982 & 0,9803  & 0,98077 & 0,98124 & 0,98169 \\
%2,1 & 0,98214 & 0,98257 & 0,983   & 0,98341 & 0,98382 & 0,98422 & 0,98461 & 0,985   & 0,98537 & 0,98574 \\
%2,2 & 0,9861  & 0,98645 & 0,98679 & 0,98713 & 0,98745 & 0,98778 & 0,98809 & 0,9884  & 0,9887  & 0,98899 \\
%2,3 & 0,98928 & 0,98956 & 0,98983 & 0,9901  & 0,99036 & 0,99061 & 0,99086 & 0,99111 & 0,99134 & 0,99158 \\
%2,4 & 0,9918  & 0,99202 & 0,99224 & 0,99245 & 0,99266 & 0,99286 & 0,99305 & 0,99324 & 0,99343 & 0,99361 \\
%2,5 & 0,99379 & 0,99396 & 0,99413 & 0,9943  & 0,99446 & 0,99461 & 0,99477 & 0,99492 & 0,99506 & 0,9952  \\
%2,6 & 0,99534 & 0,99547 & 0,9956  & 0,99573 & 0,99585 & 0,99598 & 0,99609 & 0,99621 & 0,99632 & 0,99643 \\
%2,7 & 0,99653 & 0,99664 & 0,99674 & 0,99683 & 0,99693 & 0,99702 & 0,99711 & 0,9972  & 0,99728 & 0,99736 \\
%2,8 & 0,99744 & 0,99752 & 0,9976  & 0,99767 & 0,99774 & 0,99781 & 0,99788 & 0,99795 & 0,99801 & 0,99807 \\
%2,9 & 0,99813 & 0,99819 & 0,99825 & 0,99831 & 0,99836 & 0,99841 & 0,99846 & 0,99851 & 0,99856 & 0,99861 \\
%3   & 0,99865 & 0,99869 & 0,99874 & 0,99878 & 0,99882 & 0,99886 & 0,99889 & 0,99893 & 0,99896 & 0,999   \\
%3,1 & 0,99903 & 0,99906 & 0,9991  & 0,99913 & 0,99916 & 0,99918 & 0,99921 & 0,99924 & 0,99926 & 0,99929 \\
%3,2 & 0,99931 & 0,99934 & 0,99936 & 0,99938 & 0,9994  & 0,99942 & 0,99944 & 0,99946 & 0,99948 & 0,9995  \\
%3,3 & 0,99952 & 0,99953 & 0,99955 & 0,99957 & 0,99958 & 0,9996  & 0,99961 & 0,99962 & 0,99964 & 0,99965 \\
%3,4 & 0,99966 & 0,99968 & 0,99969 & 0,9997  & 0,99971 & 0,99972 & 0,99973 & 0,99974 & 0,99975 & 0,99976 \\
%3,5 & 0,99977 & 0,99978 & 0,99978 & 0,99979 & 0,9998  & 0,99981 & 0,99981 & 0,99982 & 0,99983 & 0,99983 \\
%3,6 & 0,99984 & 0,99985 & 0,99985 & 0,99986 & 0,99986 & 0,99987 & 0,99987 & 0,99988 & 0,99988 & 0,99989 \\
%3,7 & 0,99989 & 0,9999  & 0,9999  & 0,9999  & 0,99991 & 0,99991 & 0,99992 & 0,99992 & 0,99992 & 0,99992 \\
%3,8 & 0,99993 & 0,99993 & 0,99993 & 0,99994 & 0,99994 & 0,99994 & 0,99994 & 0,99995 & 0,99995 & 0,99995 \\
%3,9 & 0,99995 & 0,99995 & 0,99996 & 0,99996 & 0,99996 & 0,99996 & 0,99996 & 0,99996 & 0,99997 & 0,99997 \\
%4   & 0,99997 & 0,99997 & 0,99997 & 0,99997 & 0,99997 & 0,99997 & 0,99998 & 0,99998 & 0,99998 & 0,99998
\end{tabular}
\end{table}




\end{landscape}

\restoregeometry

\pagebreak 

.

\end{document}

